\documentclass[11pt]{article}
%Importing custom commands
\usepackage{latex_goon/latex_goon}
\title{Linear Algebra}
\author{Garrett Goon}
\begin{document}
%
%\maketitle

\vspace{1truecm}
%
%
\renewcommand{\thefootnote}{\fnsymbol{footnote}}
\begin{center}
{\huge \bf{Linear Algebra}}
\end{center}


\begin{abstract}

Notes on various linear algebra topics that I've forgotten or never knew.

\end{abstract}

\tableofcontents


\renewcommand*{\thefootnote}{\arabic{footnote}}
\setcounter{footnote}{0}

\section{The Moore-Penrose Pseudoinverse \label{sec_mp_pseudoinverse}}

A general matrix $ M \in \mathbb{R} ^{ m\times n } $ will not have an inverse, but it is guaranteed to have a next-best-thing which is
also unique: the \textbf{Moore-Penrose Pseudoinverse} \footnote{Some concise notes on the properties
of this matrix\href{https://www.math.ucla.edu/~laub/33a.2.12s/mppseudoinverse.pdf}{can be found
here}.} , denoted by $ M ^{ + } \in \mathbb{R}^{
n\times m } $.  Obviously, it
cannot satisfy $ M M ^{ + } = \mathbf{1} $ in general (since the identity doesn't even exist for the
general $ n \neq m $ case we have in mind). However, it satisfies the closest analogues\footnote{These properties do not imply one another.} :
\begin{align}
  M ^{ + }M M ^{ + } &= M ^{ + } \nn
  M M ^{ + } M  &= M  \nn
  (M M ^{ + }) ^{ T } &= M M ^{ + } \nn
   (M ^{ + } M ) ^{ T } &= M ^{ + }M \ . \label{eq_mp_pseudoinverse}
\end{align}


In the limit where the columns are linearly-independent (such that $ M M ^{ T } $ is invertible), the MP
inverse has the explicit form $ M ^{ + } (M M ^{ T })^{ -1 } $. Similarly, if the rows are
linearly-independent (such that $ M ^{ T } M  $ is invertible), then $ M ^{ + } = (M ^{ T }M) ^{ -1
}  M ^{ T }$. By uniqueness (not proven here), $ M ^{ + } $ reduces to $ M ^{ -1 } $, when it
exists.


If solutions $ x $ of linear equations of the form $ M \cdot x = b $ exist, they can be written in terms of the
psueduoinverse:
\begin{align}
    x &= M ^{ +}b + M \left ( \mathbf{1} - M ^{ + }M \right )y \ , \quad y \ \mathrm{arbitrary}
\end{align}
which generalizes the form of the solution when $ M $ is invertible. Demonstrating that the above
solves the desired equation follows from the properties in \eqref{eq_mp_pseudoinverse}, and an
additional result that $   MM ^{ + }b = b$ necessarily holds if any solutions exist \footnote{Proof:
obviously for solutions to exist, $ b $ must live in the range of $ M $. When this is true, there is
a $ v $ for which $ Mv=b $. Then, from \eqref{eq_mp_pseudoinverse}, $b= Mv=M M ^{ + } M v= M M ^{ +
}b $.}.

\appendix


\section{Conventions and Notation}\label{app_conventions}



 \section{TODO}


 \begin{itemize}
 \item
 \end{itemize}

\bibliographystyle{latex_goon/utphys}
\bibliography{bibliography}
\end{document}
